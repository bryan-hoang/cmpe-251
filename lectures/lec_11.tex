\lecture{11}{2021-11-24}{Social media and data analysis}

Big Tech companies, like Facebook \& Google, tend to not use Data Analytics in the most ethical way\ldots

\section{Danger \#1: Selling your \textit{Personally Identifiable Information (PII)} \& attention for free stuff}

\begin{itemize}
  \item Big Tech gives you personalize ads and free stuff
  \item You give them PII and attention
  \item Repeat the cycle initiated by the Big Tech company (``stickiness''). e.g., Misinformation i prevalent since people may be seeking it out.
  \item Analogy: You are the ``\textit{Dragon}''
  \item The D.A. they use improves personalization, and eventually narrow down on \textbf{price sensitivity}. Then they sell it to everyone else
\end{itemize}

\section{Danger \#2: Social media is inherently biased against you}

\begin{itemize}
  \item People tend to post good things, not bad things
  \item The structure reinforces the bias
  \item e.g., The Friend Paradox - Homophily (ho-moph-ily),``birds of a feather flock together''.

        Insert Figure here.
\end{itemize}

\section{Previous Exam Question}

\begin{itemize}
  \item Q's will have a ``twist'', to make us think about what the actual problem to solve is. Don't just apply the 5 steps of data analytics.
  \item e.g., Baseball data sorted based on ascending salary (400 k - 2.8 mil). How does a player's performance relate to their salary, and whether it's fair? (won't be given data on exam, mostly for demonstration during lecture)
        \begin{itemize}
          \item To assess ``fairness'', consider \textbf{clustering} players to see if salaries do reflect similarly performant players. Be careful with ranking/selecting attributes.
          \item Some figuress from SVD is shown in MATLAB, although we won't have this on the exam.
          \item Seeing high performing players with low salaries amongst other high paid players implies players are being paid unfairly.
          \item In KNIME, looking at correlation between attributes for analyzing attributes. Then used EM clustering (good go to). To decide \# of clusters, try a couple (3-5). Meh results, as expected form SVD figures form earlier.
          \item Few data points (e.g., potential outliers) should be taken with a grain of salt,
          \item Conclusion: It is unfair. How to fix it? Based on analysis.
        \end{itemize}
\end{itemize}
